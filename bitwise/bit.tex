\chapter{Bit}
\section{Operadores}
\begin{lstlisting}[language=C++, keywordstyle=\color{blue}]
== -> totalmente iguais;
!= -> pelo menos um bit diferente;
<< -> right shift; >> -> left shift;
^, ^= -> XOR; &, &= -> AND; |, |= -> OR;
~ -> operador de inversao;
\end{lstlisting}

\section{Bitset}
PS: Aceita operações com operadores binários.\\
Aplicações: vetor estático de booleano otimizado.
\begin{lstlisting}[language=C++]
#define MAX 16
bitset<MAX> bs_1;// bs_1: 0000000000000000
bitset<MAX> bs_2(string("1100"));// bs_2: 0000000000001100
bitset<MAX> bs_3(18);// bs_3: 0000000000010010
bitset<4> bs; // 0000
bs[1] = 1; // 0010
bs.count(); // retorna a qtd de 1's
bs.size(); // tamanho do bitset
bs.all(); // true if all bits are 1
bs.any(); // true if at least one bit is == 1
bs.none(); // true if none all bits == 0
bs.set(); bs.set(position); bs.set(position, bit);
bs.reset(); bs.reset(position);
bs.flip(); bs.flip(position);
bs.to_string(); bs.to_string(ch); bs.to_string('bit', ch);
bs.to_ulong(); bs.to_ullong();
\end{lstlisting}