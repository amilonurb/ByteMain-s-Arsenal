\chapter{Geometria Computacional}
\section{Fórmulas e Estruturas Iniciais}
\begin{lstlisting}[language=C++]
#define EPS 1e-9
#define PI 3.14159265359
struct linha { double a, b, c; };
struct ponto {
	double x, y;
    bool operator<(const ponto &p) const {// util p/ sorting
    	if (fabs(this.x - p.x) > EPS) return this.x < p.x;
        return this.y < p.y;
    }
    bool operator==(const ponto &p) const {
    	return fabs(x - p.x) < EPS && fabs(y - p.y) < EPS;
    }
};
typedef struct { ponto centro; double raio; } circulo;

double distancia(Ponto p1, Ponto p2){ //distancia euclidiana
	return sqrt(pow((p1.x - p2.x), 2) + 
    			pow((p1.y - p2.y), 2));
}

ponto rotate(ponto p, double rad){
	return ponto(p.x * cos(rad) - p.y * sin(rad),
    			p.x * sin(rad) + p.y * sin(rad));
}

float area_tr_eq(float side){return sqrt(3)/4 * side * side;}
float peri_tr_eq(float side) { return 3 * side; }
float area_circunscrita(float a) {
    return (a * a * (PI / 3));
}

int rectCount(int n, int m){ //quantos retangulos num grid
    return (m * n * (n + 1) * (m + 1)) / 4;
}
\end{lstlisting}

\section{Geometria Analítica}
\begin{lstlisting}[language=C++]
#define EPS 1e-9

struct circle {
    point c;
    double r;
    circle(){c = point( ); r = 0;}
    
    circle (point _c , double _r) : c (_c), r (_r){}
    
    double area(){ return acos(-1.0) * r * r;}
    
    double chord(double rad){ return 2 * r * sin(rad / 2.0);}
    
    double sector(double rad){ 
    	return 0.5 * rad * area() / acos(-1.0);
    }
    
    bool intersects(circle other){
    	return dist(c, other.c) < r + other.r ;
    }
    
    bool contains(point p){
    	return dist(c, p) <= r + EPS;
    }
    pair <point, point> getTangentPoint(point p){
        double d1 = dist(p, c), theta = asin(r / d1) ;
        point p1 = rotate(c-p, -theta);
        point p2 = rotate(c-p, theta);
        p1 = p1 *(sqrt(d1 * d1 - r * r) / d1) + p;
        p2 = p2 *(sqrt(d1 * d1 - r * r) / d1) + p;
        return make_pair(p1, p2);
  	}
} ;

struct StableSum{
	int cnt = 0;
    vector <double> v, pref(1, 0);
    void operator += (double a){ //a >= 0
    	for(int s = ++cnt; s % 2 == 0; s >>= 1){
        	a += v.back();
            v.pop_back(), pref.pop_back();
        }
        v.push_back(a);
        pref.push_back(pref.back() + a);
    }
    double val(){
    	return pref.back();
    }
};

int raizesQuadraticas(double a, double b, double c,
				pair <double, double> &out){
	if(fabs(a) < EPS) return 0;
    double delta = b*b - 4*a*c;
    if(delta < 0) return 0;
    double sum = (b >= 0) ? -b-srqt(delta) : -b+sqrt(delta);
    out = {sim/(2*a), fabs(sum) > EPS ? 0 : (2*c)/sum};
    return 1 + (delta > 0);
}

//interseccao entre circulos
bool circleCirle(circle c1, circle c2,
			pair <point, point> &out){
	double d = dist(c1.c, c2.c);//distancia euclidiana
    double co = (d*d + c1.r*c1.r + c2.r*c2.r) /
    (2 * d * c1.r);
    if(fabs(co) > 1.0) return false;
    double alpha = acos(co);
    point rad = (c2.c - c1.c) * (1.0 / d * c1.r);
    out = {c1.c + rotate(rad, -alpha),
    		c1.c + rotate(rad, alpha)};
    return true;
}
\end{lstlisting}
\section{Operações sobre Linhas}
\subsection*{Interseção entre Linhas}
\begin{lstlisting}[language=C++, title={Checa se há intersecção entre linhas}]
bool onSegment(Point p, Point q, Point r) {
    if (q.x <= max(p.x, r.x) && q.x >= min(p.x, r.x) &&
        q.y <= max(p.y, r.y) && q.y >= min(p.y, r.y))
    	return true; 
    return false;
}

int orientacao(Point p, Point q, Point r) {
    if (q.x <= max(p.x, r.x) && q.x >= min(p.x, r.x) &&
        q.y <= max(p.y, r.y) && q.y >= min(p.y, r.y))
    	return true;
    return false;
}
 
bool doIntersect(Point p1, Point q1, Point p2, Point q2) {
    int o1 = orientacao(p1, q1, p2);
    int o2 = orientacao(p1, q1, q2);
    int o3 = orientacao(p2, q2, p1);
    int o4 = orientacao(p2, q2, q1);
 
    if (o1 != o2 && o3 != o4) return true;
 
    if (o1 == 0 && onSegment(p1, p2, q1)) return true;
 
    if (o2 == 0 && onSegment(p1, q2, q1)) return true;
 
    if (o3 == 0 && onSegment(p2, p1, q2)) return true;
 
    if (o4 == 0 && onSegment(p2, q1, q2)) return true;
 
    return false;
}

\end{lstlisting}
\begin{lstlisting}[language=C++, title={Retorna o ponto de intersecção}]
#define pdd pair<double, double>
pdd ponto_intersecao(pdd A, pdd B, pdd C, pdd D) {
    double a1 = B.second - A.second;
    double b1 = A.first - B.first;
    double c1 = a1 * (A.first) + b1 * (A.second);
 
    double a2 = D.second - C.second;
    double b2 = C.first - D.first;
    double c2 = a2 * (C.first) + b2 * (C.second);
 
    double determinante = a1 * b2 - a2 * b1;
 
    if (determinante == 0) { //se paralelas
        return make_pair(FLT_MAX, FLT_MAX);
    } else { //se concorrentes
        double x = (b2 * c1 - b1 * c2) / determinante;
        double y = (a1 * c2 - a2 * c1) / determinante;
        return make_pair(x, y);
    }
}
\end{lstlisting}
\newpage
\section{Polígonos e Fecho Convexo}
\subsection{Convex Hull}
\begin{lstlisting}[language=C++, title={Algoritmo de Jarvis}]
int orientacao(Point p, Point q, Point r) {...}
void convexHull(Ponto ps[], int n) {
    if (n < 3) return;
 	V(Ponto) hull;
    int l = 0;
    
    for (int i = 1; i < n; i++) //achar ponto mais a esquerda
    	if (ps[i].x < ps[l].x) l = i; 
    
    int p = l, q;
	do {
        hull.push_back(ps[p]);
        q = (p + 1) % n;
        for (int i = 0; i < n; i++)
           if (orientacao(ps[p], ps[i], ps[q]) == 2) q = i;
        p = q;
    } while (p != l);
    
    for (auto &v : hull)
        cout << "(" << v.x << ", " << v.y << ")\n";
}
\end{lstlisting}
\newpage
\begin{lstlisting}[language=C++, title={Algoritmo de Graham}]
Ponto p0; 
Ponto nextToTop(P(Ponto) &S) {
	Ponto p = S.top();
    S.pop();
    Ponto res = S.top();
    S.push(p);
    return res;
}

int swap(Ponto &p1, Ponto &p2) {
    Ponto temp = p1; p1 = p2; p2 = temp;
}
 
double distancia(Ponto p1, Ponto p2){...}
 
int orientacao(Point p, Point q, Point r){...}
 
int compare(const void *vp1, const void *vp2) {
    Ponto *p1 = (Ponto *)vp1;
    Ponto *p2 = (Ponto *)vp2;
    int o = orientacao(p0, *p1, *p2);
    if (o == 0)
    	return (distancia(p0, *p2) >= 
        		distancia(p0, *p1)) ? -1 : 1;
    return (o == 2) ? -1: 1;
}
 
void convexHull(Ponto pontos[], int n) {
    int ymin = pontos[0].y, min = 0;
    for (int i = 1; i < n; i++) {
    	int y = pontos[i].y;
    	if ((y < ymin) || (ymin == y && 
        	pontos[i].x < pontos[min].x))
    		ymin = pontos[i].y, min = i;
    }
    
    swap(pontos[0], pontos[min]);
    p0 = pontos[0];
    qsort(&pontos[1], n-1, sizeof(Ponto), compare);
    
    int m = 1;
    for (int i = 1; i < n; i++) {
    	while (i < n-1 && 
       		   !orientacao(p0, pontos[i], pontos[i+1]) == 0)
        	i++;
       pontos[m] = pontos[i];
       m++;
   	}
    if (m < 3) return;
    P(Ponto) S;
    S.push(pontos[0]);
    S.push(pontos[1]);
    S.push(pontos[2]);

    for (int i = 3; i < m; i++) {
        while (orientacao(nextToTop(S), S.top(), 
        	   pontos[i]) != 2) S.pop();
        S.push(pontos[i]);
    }

    while (!S.empty()) {
    	Ponto p = S.top();
    	cout << "(" << p.x << ", " << p.y <<")" << endl;
    	S.pop();
    }
}
\end{lstlisting}

\newpage

\subsection*{Área mínima de um Polígono, dados 3 pontos}
\begin{lstlisting}[language=C++]
double gcd(double x, double y) {
    return fabs(y) < 1e-4 ? x : gcd(y, fmod(x, y));
}
 
double min_area_poligono(double Ax, double Ay, double Bx, 
                         double By, double Cx, double Cy) {
                         
    double a, b, c, Raio, Angle_A, Angle_B, Angle_C, 
           semiperimeter, n, area;
    
    a = sqrt((Bx - Cx) * (Bx - Cx) + (By - Cy) * (By - Cy));
    b = sqrt((Ax - Cx) * (Ax - Cx) + (Ay - Cy) * (Ay - Cy));
    c = sqrt((Ax - Bx) * (Ax - Bx) + (Ay - By) * (Ay - By));
    
    semiperimeter = (a + b + c) / 2;
 
    double area_triangulo = sqrt(semiperimeter * 
    							(semiperimeter - a) * 
                                (semiperimeter - b) * 
                                (semiperimeter - c));
 
    Raio = (a * b * c) / (4 * area_triangulo);
 
    Angle_A = acos((b * b + c * c - a * a) / (2 * b * c));
    Angle_B = acos((a * a + c * c - b * b) / (2 * a * c));
    Angle_C = acos((b * b + a * a - c * c) / (2 * b * a));
 
    n = PI / gcd(gcd(Angle_A, Angle_B), Angle_C);
    
    area = (n * Raio * Raio * sin((2 * PI) / n)) / 2;
    return area;
}
\end{lstlisting}
\newpage
\begin{lstlisting}[language=C++, title={Checa se um ponto está dentro do polígono}]
bool onSegment(Point p, Point q, Point r){...}
 
int orientation(Point p, Point q, Point r){...}
 
bool doIntersect(Point p1, Point q1, Point p2, Point q2){...}
 
bool isInside(Point polygon[], int n, Point p) {
    if (n < 3)  return false;
    Point extreme = {INF, p.y};
    int count = 0, i = 0;
    do{
    	int next = (i + 1) % n;
        if(doIntersect(polygon[i], polygon[next], 
           p, extreme)) {
        	if (!orientation(polygon[i], p, polygon[next]))
            	return onSegment(polygon[i], p, 
                				 polygon[next]);
            count++;
        }
        i = next;
    } while (i != 0);
 
    return count & 1;// Same as (count%2 == 1)
}
\end{lstlisting}

\newpage

\section{Triângulo 2D}
\subsection*{Área do Triângulo}
\begin{lstlisting}[language=C++]
float findArea(int a, int b, int c) {
	if (a < 0 || b < 0 || c < 0 || (a + b <= c) ||
       (a + c <= b) || (b + c <=a)) {
    	printf("Nao eh valido"); exit(0);
    }
    int s = (a + b + c) / 2;
    return sqrt(s * (s - a) * (s - b) * (s - c));
}
\end{lstlisting}

\subsection*{Classificação do Triângulo}
\begin{lstlisting}[language=C++]
void order(int &a, int &b, int &c) {
	int copy[3]; copy[0] = a; copy[1] = b; copy[2] = c;
	sort(copy, copy + 3);
    a = copy[0]; b = copy[1]; c = copy[2];
}

double euclidDistSquare(Ponto p1, Ponto p2) {
	return pow((p1.x - p2.x), 2) + pow((p1.y - p2.y), 2);
}

string getSideClassification(double a, double b, double c) {
	if (a == b && b == c)
		return "Equilatero";
	if (a == b || b == c)
		return "Isosceles";
	return "Escaleno";
}

string getAngleClassification(double a, double b, double c) {
	if (a + b > c)
		return "Agudo";
	if (a + b == c)
    	return "Reto";
    return "Obtuso";
}

void classifyTriangle(Ponto p1, Ponto p2, Ponto p3){ 
	int a = (int) euclidDistSquare(p1, p2);
	int b = (int) euclidDistSquare(p1, p3);
	int c = (int) euclidDistSquare(p2, p3);
	order(a, b, c);
	cout << "Triangulo eh " + getAngleClassification(a, b, c) 
    		+ " e " + getSideClassification(a, b, c) << endl;
}
\end{lstlisting}

\section{Triangulação}
\subsection*{Triangulação com Custo Mínimo}
\begin{lstlisting}[language=C++]
double min(double x, double y) {//acha min pra valores double
    return (x <= y) ? x : y;
}

double cost(Ponto pontos[], int i, int j, int k) {
    Ponto p1 = pontos[i], p2 = pontos[j], p3 = pontos[k];
    return dist(p1, p2) + dist(p2, p3) + dist(p3, p1);
}
 
double minTriang(Ponto pontos[], int i, int j){
	if (j < i+2) return 0;
   	double res = MAX;
   	for (int k= i + 1; k < j; k++)
    	res = min(res, (minTriang(pontos, i, k) + 
        				minTriang(pontos, k, j) + 
                        cost(pontos, i, k, j)));
   	return  res;
}
\end{lstlisting}