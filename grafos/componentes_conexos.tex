\begin{lstlisting}[language=C++]
m(int) grafo;
int connectedComponents(int num_vertices) {
	v(int) elementos;
	int status[num_vertices], init = 0, pilha[num_vertices], 
	    topo = -1, result = 0;
    memset(status, 0, sizeof(status));
	rep(j, num_vertices) {
		if (status[j] == 0) {
			status[j] = 1;
			pilha[++topo] = j;
			while (topo != -1) {
				init = pilha[topo--];
				elementos.pb(init);
				rep(i, num_vertices) {
					if (grafo[init][i] && status[i] == 0) {
						pilha[++topo] = i;
						status[i] = 1;
					}
				}
			}
			// Rotina para exibir as componentes
            // sort(elementos.begin(), elementos.end());
			// rep(i, elementos.size())
			//    printf("%n ", elementos[i]);
			// putchar('\n');
            elementos.clear();
			result++;
		}
	}
	return result;
}
\end{lstlisting}