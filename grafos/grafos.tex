\chapter{Grafos}

\section{Representações}
\subsection*{Matriz de Adjacências}

\begin{lstlisting}[language=C++, title={Matriz de Adjacências}]
#define M(x) vector<vector<int>>
M(x) grafo;
\end{lstlisting}

\begin{lstlisting}[language=C++, title={Lista de Adjacências}]

\end{lstlisting}

\section{DFS}
\subsection*{Matriz de Incidência - Beta}
\begin{lstlisting}[language=C++]
m(int) grafo; //matriz de incidencia
v(int) val; //pesos
v(bool) visitado; //aresta nao visitada
s(int) caminho;
int n, m;

void init(int n, m){
	for(int i=0; i < n; i++) 
		grafo.pub(vector<int>(m,0));
}

void add(int x, int y, int e, int val){
	grafo[x][e] = origem;
	grafo[y][e] = destino;
}

int getFilho(int pai){
	for(int i = 0; i < m; i++){
		if(grafo[pai][i] == origem and not visitado[i]){
			for(int j=0; j <n; j++ ){
				if(grafo[i][j] == destino){
					visitado[i]=true;
					return j;
				}
			}
		}
	}
	return nill;
}

void dfs(int atual){
	int filho = getFilho(atual);// O(n*m) => O(n^2)
	
	if(filho != nill){
		caminho.push(filho);
		printf("%d %d\n", pai, filho);
		dfs(filho);
	}else if(not caminho.empty()){
		int pai = caminho.top();
		caminho.pop();
		printf("%d %d\n", filho, pai);
		dfs(pai);
	}
}
\end{lstlisting}
\subsection*{Matriz de Adjacência}
\begin{lstlisting}[language=C++]
M(int) grafo; V(bool) visitado; P(int) caminho;

int filho(int vPai) {
	for (int cont = 0; cont < grafo.size(); cont++)
        if (grafo[vPai][cont] && !visitado[cont]) return cont;
    return NILL;
}

void dfs(int vAtual) {
    int vFilho = filho(vAtual); visitado[vAtual] = T;
    if (vFilho != NILL) {
        caminho.push(vAtual); dfs(vFilho);
    } else if (!caminho.empty()) {
        int vPai = caminho.top(); caminho.pop(); dfs(vPai);
    }
}
\end{lstlisting}

\subsection*{Lista de Adjacência}
\begin{lstlisting}[language=C++]
//Implementar
\end{lstlisting}

\subsection*{Grid}
\begin{lstlisting}[language=C++]
//Implementar
\end{lstlisting}

\newpage

\section{BFS}
\subsection*{Matriz de Adjacência}
\begin{lstlisting}[language = C++]
//Implementar
\end{lstlisting}

\section{BFS - Lista de Adjacência}
\begin{lstlisting}[language=C++]
//Implementar
\end{lstlisting}

\section{BFS - Grid}
\begin{lstlisting}[language=C++]
//Implementar
\end{lstlisting}

\newpage

\section{Flood-Fill}
\begin{lstlisting}[language = C++]
M(bool) grid; V(int) tamanhos; int tamanho=0;

void dfs(int x, int y) {
	if((x >= 0 && x <= grid.size()) && 
       (y >= 0 && y <= grid[0].size())) {
    	if(!grid[x][y]) {
            tamanho++; //tamanho da componente
            grid[x][y] = true;
            // cruz
            dfs(x-1, y); dfs(x+1, y);
            dfs(x, y+1); dfs(x, y-1);
            //diagonais
            dfs(x-1, y+1); dfs(x+1, y+1);
            dfs(x+1, y-1); fs(x-1, y-1);
        }
    }
}

void floodFill() {
    for (int x = 0; x < grid.size(); x++) {
    	for (int y = 0; y < grid[0].size(); y++) {
    		if(!grid[x][y]) {
            	tamanho = 0;
    			dfs(x, y);
                tamanhos.PUB(tamanho);
                // tamanhos.size() == quantidade componentes
    		}
    	}
    }
}
\end{lstlisting}

\newpage
\section{Árvore Espalhada Mínima}
\begin{lstlisting}[language=C++, title={Algoritmo de Kruskal}]
struct aresta { int x, y, z; };
V(aresta) grafo;
V(int) pai;

tuple(V(aresta),int) arvoreGeradora(int qtdVertices) {
	int peso = NILL;
    V(aresta) mst;
	pai = V(int)(quantVertice, NILL);
    
    //verifique a possibilidade de stable_sort()
    sort(grafo.begin(), grafo.end(), 
    	[](aresta a, aresta b) -> bool {
    	// return a.z > b.z // maxima
        return a.z < b.z; // minima
    });
    
	for(aresta it : grafo) {
		if(!formaCiclo(it.x, it.y)) {
			_union(it.x, it.y);
			peso += it.z;
            mst.PUB(it);
		}
	}
	return (mst,peso);
}
\end{lstlisting}

\newpage

\section{Caminhos Mínimos}


\section{Componentes Conexos}
\begin{lstlisting}[language=C++, title={Versão 1}]
// v: qtd de vertices
int componentesConexos(M(int) grafo, int v) {
	V(int) elementos; int status[v], pilha[v];
    int init = 0, topo = -1, result = 0, i, j;
    
    memset(status, 0, sizeof(status));
	// for (i = 0; i < v; i++) status[i] = 0;
	
    for (j = 0; j < v; j++)	{
		if (status[j] == 0) {
			status[j] = 1;
			pilha[++topo] = j;

			while (topo != -1) {
				init = pilha[topo--];
				elementos.PUB(init);
				
				for (i = 0; i < v; i++)	{
					if (grafo[init][i] && status[i] == 0) {
						pilha[++topo] = i;
						status[i] = 1;
					}
				}
			}

			// se for exibir os elementos de cada componente
			sort(elementos.begin(), elementos.end());
            for (auto &e : elementos) cout << e << ' ';
			putchar('\n');

			elementos.clear(); result++;
		}
	}

	return result;
}
\end{lstlisting}

\begin{lstlisting}[language=C++, title={v2}]
// mesma ideia de floodfill
for(int cont=0; cont < visitado.size(); cont++ ){
	if(!visitado[cont]){
		dfs(cont);
		componente++;
	}
}
\end{lstlisting}

\newpage

\section{Emparelhamento Máximo (Grafo Bipartido)}
\begin{lstlisting}[language=C++]
// TROCAR "V" POR "V(int)" E "M" por "const M(int)"
bool findMatch(int i, M &w, V &mr, V &mc, V &seen) {
    for (int j = 0; j < w[i].size(); j++) {
    	if (w[i][j] && !seen[j]) {
    		seen[j] = true;
            if (mc[j] < 0 || 
            	findMatch(mc[j], w, mr, mc, seen)) {
                mr[i] = j; mc[j] = i;
                return true;
            }
    	}
    }
    return false;
}
// TROCAR "V" POR "V(int)" E "M" por "const M(int)"
int bipartiteMatching(M &w, V &mr, V &mc) {
    mr = V(int)(w.size(), -1);
    mc = V(int)(w[0].size(), -1);
    int ct = 0;
    for (int i = 0; i < w.size(); i++) {
    	V(int) seen(w[0].size());
    	if (findMatch(i, w, mr, mc, seen)) ct++;
    }
    return ct;
}
\end{lstlisting}