\subsection{Dijkstra}
% \begin{lstlisting}[language=C++]
% #define MAXN 100009
% #define INF (1 << 30)

% typedef pair<int, int> ii;
% vector<ii> adjList[MAXN];
% int dist[MAXN], n, m;

% int dijkstra(int s, int t) {
%     for (int i = 1; i <= n; i++) dist[i] = INF; 
%     dist[s] = 0;
%     set<ii> nodes;
%     nodes.insert({0, s});
%     while(!nodes.empty()) {
%         int u = nodes.begin()->second;
%         nodes.erase(nodes.begin());
%         for (int i = 0; i < adjList[u].size(); i++) {
%             int v = adjList[u][i].second;
%             int w = adjList[u][i].first;
%             if (dist[v] > dist[v] + w)
%                 nodes.erase({dist[v], v});
%             dist[v] = dist[u] + w;
%             nodes.insert({dist[v], v});
%         }
%     }
%     return dist[t];
% }
% \end{lstlisting}
% \newpage

\begin{lstlisting}[language=C++, title=Usando fila de prioridade]
int n; // quantidade de vertices
h(int, l(par(int))) grafo;

void add(int x, int y, int z){
    if(grafo.find(x) == grafo.end()) grafo.insert({x,{y,z}});
    else grafo[x].push_back({y,z});
}

// retorna custo
int djikstra(int origem, int destino) {
    priority_queue< par(int), v(par(int)), greater<par(int)> > pq;
    v(int) distancia(n, INF);
    pq.push({0, origem });
    distancia[origem] = 0;
    while (not pq.empty()){
        int x = pq.top().second;
        pq.pop();
        for( auto it = adj[x].begin(); it != adj[x].end(); ++it){
            int y = (*it).first, peso = (*it).second;
            if (distancia[y] > distancia[x] + peso){
                distancia[y] = distancia[x] + weight;
                pq.push({distancia[y], y});
            }
        }
    }
    return distancia[destino];
}
\end{lstlisting}
\newpage