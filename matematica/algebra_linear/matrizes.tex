\begin{lstlisting}[language=C++]
#define MOD 1000000007
typedef m(int) matrix;

matrix matrixUnit(int n) {
    matrix identity(n, v(int)(n));
    for (int i = 0; i < n; i++) identity[i][i] = 1;
    return identity;
}

matrix operator+(const matrix &a, const matrix &b) {
    int n = (int) a.size();
    int m = (int) a[0].size();
    matrix c; c.resize(n);
    for (int i = 0; i < n; i++) {
        c[i].resize(m);
        for (int j = 0; j < m; j++)
            c[i][j] = (a[i][j] + b[i][j]) % MOD;
    }
    return c;
}

matrix operator*(const matrix &a, const matrix &b) {
    int n = (int) a.size();
    int m = (int) b.size();
    int p = (int) b[0].size();
    matrix c; c.resize(n);
    for (int i = 0; i < n; i++) {
        c[i].assign(p, 0);
        for (int j = 0; j < p; j++)
            for (int k = 0; k < m; k++)
                c[i][j] = (c[i][j] + (a[i][k] * 
                                      b[k][j]) % MOD) % MOD;
    }
    return c;
}
\end{lstlisting}

\newpage

\begin{lstlisting}[language=C++, title=Operações com Matrizes (cont.)]
matrix operator*(int k, const matrix &a) {
    int n = (int) a.size();
    int m = (int) a[0].size();
    matrix b;
    b.resize(n);
    for (int i = 0; i < n; i++) {
        b[i].assign(m, 0);
        for(int j = 0; j < m; j++)
            b[i][j] = (a[i][j] * k) % MOD;
    }
    return b;
}

matrix operator-(const matrix &a, const matrix &b) {
    return a + ((-1) * b);// se for m(double), mudar p/ -1.0
}

matrix matrixExp(const matrix &a, int n) {
    if (n == 0) return matrixUnit(a.size());
    matrix b = matrixExp(a, n / 2);
    b = b * b;
    if (n % 2 != 0) b = b * a;
    return b;
}
\end{lstlisting}