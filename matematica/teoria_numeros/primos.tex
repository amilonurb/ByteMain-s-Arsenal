\subsubsection{Testes de Primalidade}
\begin{lstlisting}[language=C++]
bool is_prime(big n) {
    if (n < 0) return is_prime(n * (-1));
    for (big i = 2; i * i < n; i++)
        if (n % i == 0) return false;
    return true;
}

bool is_prime_fast(big n) {
    if (n < 0) n = n * (-1);
    if (n < 5 || n % 2 == 0 || n % 3 == 0)
        return (n == 2 || n == 3);
    big limit = sqrt(n) + 2;
    for (big p = 5; p < limit; p += 6) {
        if (p < n && n % p == 0) return false;
        if ((p + 2) < n && n % (p + 2) == 0) return false;
    }
    return true; }
\end{lstlisting}

\subsubsection{Crivo de Eratóstenes}
\begin{lstlisting}[language=C++]
big sieve_size; bs(MAX) isp; v(big) primes;
void sieve(big n) {
    sieve_size = n + 1; isp.set(); isp[0] = isp[1] = 0;
    for (big i = 2; i <= sieve_size; i++) {
        if (isp[i])
            for (big j = (i * i); j <= sieve_size; j += i)
            isp[j] = 0;
        primes.pb(i); } }

bool is_prime_sieve(big n) {
    if (n <= sieve_size) return isp[n];
    for (int i = 0; (i < primes.size()) && 
                    (primes[i] * primes[i] <= n); i++) {
        if (n % primes[i] == 0) return false; }
    return true; }
\end{lstlisting}

\newpage

\subsubsection{Crivo de Atkin}
\begin{lstlisting}[language=C++]
v(int) atkinSieve(int limit) {
    v(int) sieve;
    if (limit > 2) sieve.pb(2);
    if (limit > 3) sieve.pb(3);
    v(bool) is_prime(limit + 1, false);
    for (int x = 1; x * x < limit; x++) {
        for (int y = 1; y * y < limit; y++) {
            // Main part of Sieve of Atkin
            int n = (4 * x * x) + (y * y);
            if (n <= limit && (n % 12 == 1 || n % 12 == 5))
                is_prime[n] = (is_prime[n] ^ true);
            n = (3 * x * x) + (y * y);
            if (n <= limit && n % 12 == 7)
                is_prime[n] = (is_prime[n] ^ true);
            n = (3 * x * x) - (y * y);
            if (x > y && n <= limit && n % 12 == 11)
                is_prime[n] = (is_prime[n] ^ true);
        }
    }

    // Mark all multiples of squares as non-prime
    for (int r = 5; r * r < limit; r++) {
        if (is_prime[r])
            for (int i = r * r; i < limit; i += r * r)
                is_prime[i] = false;
    }

    for (int a = 5; a <= limit; a++)
        if (is_prime[a]) sieve.pb(a);
    
    return sieve;
}
\end{lstlisting}

\newpage

\subsubsection{Fatoração de Primos}
\begin{lstlisting}[language=C++]
v(int) primeFactors(big n) {// primes gerado por Crivo
    v(int) factors; big index = 0, pf = primes[index];
    while ((pf * pf) <= n) {
        while (n % pf == 0) { n /= pf; factors.pb(pf); }
        pf = primes[++index]; }
    if (n != 1) { factors.pb(n); } return factors;
}

big number_of_divisors(big n) {
	big i = 0, p = primes[i], answer = 1;
	while (p * p <= n) {
		big power = 0;
		while (n % p == 0) { n /= p; power++; }
		answer *= (power + 1); p = primes[++i];
	}
	if (n != 1) { answer *= 2; } return answer;
}

big numDiffPF[MAXN];// num de fatores primos distintos
void genDiffPF() {
	memset(numDiffPF, 0, sizeof(numDiffPF));
	for (int i = 2; i < MAXN; i++) {
		if (numDiffPF[i] == 0)
			for (int j = i; j < MAXN; j += i) numDiffPF[j]++;
	}
}
\end{lstlisting}