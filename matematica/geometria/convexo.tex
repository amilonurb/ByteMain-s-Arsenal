Sobre a Versão 1: caso precise considerar os pontos no meio de uma aresta, trocar o teste $ccw$ para $\geq 0$.\\
CUIDADO: Se todos os pontos forem colineares, vai dar RTE.
\begin{lstlisting}[language=C++, title=Fecho Convexo - Versão 1]
point pivot(0, 0);
v(point) convexHull(v(point) P) {
	int i, j, n = (int) P.size(); if (n <= 2) return P;
	int P0 = leftmostIndex(P); swap(P[0], P[P0]);
	pivot = P[0];
	sort(++P.begin(), P.end(), angleCmp);
	v(point) S; S.pub(P[n - 1]); S.pub(P[0]); S.pub(P[1]);
	for (i = 2; i < n;) {
		j = (int) S.size() - 1;
		if (ccw(S[j - 1], S[j], P[i])) S.pub(P[i++]);
		else S.pob();
	}
	reverse(all(S)); S.pop_back(); reverse(all(s));
	return S;
}
\end{lstlisting}

\begin{lstlisting}[language=C++, title=Fecho Convexo - Versão 2 - Andrew's Monotone Chain]
v(ponto) convexHullMC(v(ponto) poligono) {
    size_t n = poligono.size(), k = 0;
	if (n <= 3) return poligono;
	vector<ponto> hull(2 * n); sort(all(poligono));
	// Build lower hull
	for (size_t i = 0; i < n; i++) {
		while (k >= 2 && cross(hull[k - 2], hull[k - 1],
		                       poligono[i]) <= 0) k--;
		hull[k++] = poligono[i]; }
	// Build upper hull
	for (size_t i = n - 1, t = k + 1; i > 0; i--) {
		while (k >= t && cross(hull[k - 2], hull[k - 1],
		                       poligono[i - 1]) <= 0) k--;
		hull[k++] = poligono[i - 1]; }
	hull.resize(k - 1);
	return hull; }
\end{lstlisting}