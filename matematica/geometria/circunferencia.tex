\begin{lstlisting}[language=C++, title=Parte I de II]
struct circle {
	point p; double raio;
	circle() { p = point(); raio = 0; }
	circle(point p, double raio): p(p), raio(raio) {}
	
	double area() { return acos(-1.0) * raio * raio; }
	
	double chord(double rad) {
	    return 2 * raio * sin(rad / 2.0); }
	    
	double sector(double rad) {
	    return 0.5 * rad * area() / acos(-1.0); }
	    
	bool intersects(circle other) {
		return dist(p, other.p) < raio + other.raio; }
		
	bool contains(point p) {
		return dist(this->p, p) <= raio + EPS; }
		
	// se o ponto estiver dentro asin retorna NAN
	pair<point, point> getTangentPoint(point p) {
		double d1 = dist(p, this->p);
		double theta = asin(raio / d1);
		point p1 = rotate(this->p - p, -theta);
		point p2 = rotate(this->p - p, theta);
		p1 = p1 * (sqrt(d1 * d1 - raio * raio) / d1) + p;
		p2 = p2 * (sqrt(d1 * d1 - raio * raio) / d1) + p;
		return make_pair(p1, p2);
	}
};
\end{lstlisting}

\newpage

\begin{lstlisting}[language=C++, title=Parte II de II]
circle circumcircle(point a, point b, point c) {
	circle answer;
	point u = point((b - a).y, -(b - a).x);
	point v = point((c - a).y, -(c - a).x);
	point n = (c - b) * 0.5;
	double t = crossProduct(u, n) / crossProduct(v, u);
	answer.p = ((a + c) * 0.5) + (v * t);
	answer.raio = dist(answer.p, a);
	return answer;
}

int insideCircle(point p, circle c) {
	if (fabs(dist(p , c.p) - c.raio) < EPS) return 1;//border
	if (dist(p , c.p) < c.raio) return 0;//inside
	return 2;//outside
}

// divisao por zero se os pontos forem colineares
circle incircle(point p1, point p2, point p3) {
    double m1 = dist(p2, p3);
    double m2 = dist(p1, p3);
    double m3 = dist(p1, p2);
    point c = (p1 * m1 + p2 * m2 + p3 * m3) * 
              (1 / (m1 + m2 + m3));
    double s = 0.5 * (m1 + m2 + m3);
    double r = sqrt(s * (s - m1) * (s - m2) * (s - m3)) / s;
    return circle(c, r);
}
\end{lstlisting}